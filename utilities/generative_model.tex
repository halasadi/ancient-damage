\documentclass[12pt]{article}
\usepackage[utf8x]{inputenc}
\usepackage{amsmath}
\usepackage{graphicx}
\usepackage[margin=1in]{geometry}
\DeclareSymbolFont{extraup}{U}{zavm}{m}{n}
\DeclareMathSymbol{\varheart}{\mathalpha}{extraup}{86}
\DeclareMathSymbol{\vardiamond}{\mathalpha}{extraup}{87}
\usepackage{float}


\begin{document}

\section{Generative Damage model for ancient DNA}

This is a summary of the model that John Novembre proposed as a generative model for ancient DNA on which inference can be carried out. 

Say for individual $i$, $\pi_i$ is the probability that it is ancient. so, if we observe a read $r$ from the $i$ th individual, then for that read, we can define a latent variable $Z_{r,i}$ such that

$$  Z_{r,i} | \pi_{i} \sim Bern (\pi_{i}) $$

Then for each position $p$ of that read (from the end of the read), we define another latent random variable 

\begin{align}
D_{r,p,i}  & = 1  \;\; \text{if damage / error } \\
	      & = 0 \;\; \text{polymorphism} \\
\end{align}

Then we can assign the following probabilities of $D | Z$.

$$ Pr \left [ D_{r, p, i} = 0  |   Z_{r,i} = 0  \right ]  =  \theta_{\pi} = 1/1000 $$
$$ Pr \left [ D_{r, p, i} = 0  |   Z_{r,i} = 1  \right ]  =  \theta_{\pi} = 1/1000 $$
$$ Pr \left [ D_{r, p, i} = 1  |   Z_{r,i} = 1  \right ]  =  f_{1} (p, \alpha) $$
$$ Pr \left [ D_{r, p, i} = 0  |   Z_{r,i} = 1  \right ]  =  f_2 (p) $$

where $f_1$ and $f_2$ are both decreasing functions with respect to $p$ (which is the minimum distance from the end of the read), we sort of assume that we know $f_2$ from 1000 genomes data may be, because otherwise it would be hard to distinguish between $f_1$ (damage) and $f_2$ (sequencing error).

From here on, we can go in two directions. If the primary aim is to detect contamination in the sample, we can then say 

$$  X_{r, p, i} | D_{r, p, i}  \sim Mult \left (1, p_{i1}, p_{i2}, \cdots, p_{ij}  \right) $$

where $$ p_{ij} = \sum_{k=1}^{K} \omega_{ik} \theta_{kj}  $$.

where $j$ is the signature pattern.

Another direction would be to define another latent random variable $M_{r,p,i}$ such that 

$$  Pr \left [ M_{r, p, i} = 0  | D_{r, p, i} = 0 \right] = 1 $$

$$ Pr \left [ M_{r, p, i} =  1 | D_{r, p, i} = 1 \right] =1 $$

$$ Pr \left[M_{r, p, i}  = 2 | D_{r, p, i} = 0 \right] = \omega $$
$$ Pr \left[M_{r, p, i}  = 2 | D_{r, p, i} = 1 \right] = 1 - \omega $$

\end{document}
